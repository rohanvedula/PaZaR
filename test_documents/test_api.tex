\documentclass{article}%
\usepackage[T1]{fontenc}%
\usepackage[utf8]{inputenc}%
\usepackage{lmodern}%
\usepackage{textcomp}%
\usepackage{lastpage}%
\usepackage{amssymb}%
\usepackage{amsfonts}%
\usepackage{amsmath}%
%
\title{My Math Notes}%
\date{\today}%
%
\begin{document}%
\normalsize%
\maketitle%
Because we previously determined that the quadratic formula has no real solutions, the denominator of the first term can never be zero, so the derivative is defined everywhere. Evaluate the endpoints and critical points: %
$g(0)$%
 %
$=$%
 In %
$(8)$%
 %
$\approx$%
 %
$2.08$%
 %
$g(6)$%
 %
$=$%
 In %
$(36$%
 %
$—$%
 %
$30$%
 %
$+$%
 %
$8)$%
 %
$=$%
 In %
$(14)$%
 %
$\approx$%
 %
$2.64$%
 %
$g(2.5)$%
 %
$=$%
 In %
$(1.75)$%
 %
$\approx$%
 %
$0.56$%
 We conclude that %
$g(6)$%
 %
$2.64$%
 is the global maximum and %
$9(2.5)$%
 %
$\approx$%
 %
$0.56$%
 Is the global minimum. %
\end{document}